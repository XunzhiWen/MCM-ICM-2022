\documentclass{mcmthesis}
\mcmsetup{CTeX = true,   % 使用 CTeX 套装时,设置为 true
        tcn = XJ162, problem = A,
        sheet = true, titleinsheet = true, keywordsinsheet = true,
        titlepage = false, abstract = false}
\geometry{left=1in,right=0.75in,top=1in,bottom=1in}
\numberwithin{figure}{section}
\numberwithin{table}{section}
\numberwithin{equation}{section}
\usepackage{newtxtext}
\usepackage{lipsum}
\usepackage{palatino}
\usepackage{hyperref}
\usepackage{booktabs}
\usepackage{subfigure}
\usepackage{graphicx}
\usepackage{pythonhighlight}

\title{test}
\setlength{\headheight}{15pt}
\begin{document}

\begin{abstract}
  Finless porpoise is the only freshwater mammal in the Yangtze River, 
  which is distributed in the middle and lower reaches of the Yangtze 
  River, Dongting Lake and Poyang Lake, and its population has 
  decreased dramatically in the past 20 years. In 1991, the number of 
  finless porpoises in the Yangtze River was more than 2,700. However,  
  in the year of 2006, there were fewer than 1,800 finless porpoises. 
  In 2011, there were probably just over 1,000 of them, 
  and in 2018 there were about 1,012. In fact, since the 1980s, 
  three conservation strategies have been gradually explored: 
  in situ conservation, ex situ conservation and artificial breeding. 
  Among them, ex situ protection, that is, selecting some waters with 
  similar ecological environment to the Yangtze River to establish 
  ex situ protection, is the most direct and effective measures 
  to protect the Yangtze finless porpoise. So far, China has set up 
  five ex-situ protected sites, with a total of more than 150 
  ex-situ populations. On September 18, 2021, CCTV reported that 
  the population of the Yangtze finless porpoise is growing steadily. 
  The population decline of the Yangtze finless porpoise has been 
  curbed, but its critically endangered status remains unchanged.


\begin{keywords}
123456
\end{keywords}
\end{abstract}
\maketitle

\tableofcontents
  \thispagestyle{empty}
  \newpage
  \setcounter{page}{1}
%%
%%Generate the Memorandum, if it's needed.
%\memoto{\LaTeX{}studio}
%\memofrom{Liam Huang}
%\memosubject{Happy \TeX{}ing!}
%\memodate{\today}
%%\logo{\LARGE I'm pretending to be a LOGO!}
%\begin{memo}[Memorandum]
%  \lipsum[1-3]
%\end{memo}

\section{Introduction}

\subsection{Problem Restatement}



\subsection{Overview of Our Work}




\section{Assumptions and Justifications}



\section{Notations}

\renewcommand\arraystretch{1.5}

\begin{table}[h]
  \centering
  \caption{Notation Descriptions} \label{Varibles}
  \begin{tabular}{m{2.5cm}<{\centering}|m{12.5cm}<{\centering}}
  \toprule[1.5pt]
  \textbf{Symbol} & \textbf{Definition} \\ \hline
  $\mathbf{A}$ & A set of artists given in dataset \\
  $\mathbf{G}$ & A set of genres provided in dataset \\
  $f_i$ & The total number of followers of artist $i$, $i\in \mathbf{A}$ \\
  $g_{ij}$ & Genre tag between artist $i$ and his or her follower $j$, $i,j \in \mathbf{A}$ \\
  $DAS_{i}$ & Artist $i$'s decade of active start, accurate to 10 years \\
  $r_{ij}$ & Respective Influence of influencer $i$ over follower $j$, $i,j \in \mathbf{A}$ \\
  $w_{i}$ & Artist $i$'s weight of normalized indexes \\ 
  $TI_{i} $ & Artist $i$'s Total Influence \\
  $wf_j$ & The parameter of follower $j$' influence, $j\in \mathbf{A}$ \\
  $wt_{i}$ & The weight of artist $i$'s Total Influence, $i,j\in \mathbf{A}$\\
  $cg_{ik}$ & Artist $i$'s Contemporary Influence in certain genre, $i\in \mathbf{A}, k \in \mathbf{G}$ \\
  $c_{i}$ & Artist $i$'s Contemporary Influence, $i\in \mathbf{A}$ \\
  $S_{ij} $ & Similarity between artists $i$ and $j$ \\
  \bottomrule[1.5pt]
  \end{tabular}
  \end{table}




\section{Model I:Directed Network of Musical Influence Model}

\section{Sensitivity Test}

\section{Evaluation of Model}

\section{Conclusions}

\newpage
\phantomsection\addcontentsline{toc}{section}{Report}\tolerance=500
\memoto{ICM society}
\memofrom{ICM Team 2104997}
\memodate{\today}

\begin{memo}[report]
  
\end{memo}




\newpage

%这一行是用来将Reference添加到目录的
\phantomsection\addcontentsline{toc}{section}{Refence}\tolerance=500

\begin{thebibliography}{99}
\bibitem[1]{bib:1} ZHANG Xianfeng, and Wang Kexiong. "Population viability analysis for the Yangtze finless porpoise." Bulletin of the Chinese Academy of Sciences:English Edition 27.1(1999):3473-3484.

\end{thebibliography}


\newpage


\lhead{\small\sffamily \team}
\rhead{\small\sffamily Page \thepage\ }

\begin{appendices}




\textbf{\textcolor[rgb]{0.98,0.00,0.00}{Input matlab source:}}
% \lstinputlisting[language=Matlab]{./similarity_of_genres.m}
% \lstinputlisting[language=Matlab]{./distance_of_characteristics_in_groups.m}
% \lstinputlisting[language=Matlab]{./revolution_artist.m}


\end{appendices}


\end{document}

